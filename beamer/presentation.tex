\documentclass{beamer}
\usepackage{tikz-cd, quiver}

\usepackage{wasysym}

\usepackage[utf8]{inputenc}

\usepackage{tikz}
\usetikzlibrary{arrows,trees,positioning,shapes,patterns}
\usetikzlibrary{intersections,calc,fpu,decorations.pathreplacing}

\usepackage[T1]{fontenc} % better fonts

% stuff from liam's prelude file
\newcommand{\no}{\noindent}
\newcommand{\vs}{\vspace{1\baselineskip}} %add one vertical space

\newcommand{\R}{\mathbb{R}}
\newcommand{\Z}{\mathbb{Z}}
\newcommand{\N}{\mathbb{N}}
\newcommand{\Q}{\mathbb{Q}}

\DeclareMathOperator{\im}{Im}

\newcommand{\inv}{^{-1}}
\newcommand{\id}{\text{id}}
\renewcommand{\hom}{\text{Hom}}
\renewcommand{\phi}{\varphi}

\newcommand{\inj}{\hookrightarrow}
\newcommand{\surj}{\twoheadrightarrow}


% Haskell code listings in our own style
\usepackage{listings,color}
\definecolor{lightgrey}{gray}{0.35}
\definecolor{darkgrey}{gray}{0.20}
\definecolor{lightestyellow}{rgb}{1,1,0.92}
\definecolor{dkgreen}{rgb}{0,.2,0}
\definecolor{dkblue}{rgb}{0,0,.2}
\definecolor{dkyellow}{cmyk}{0,0,.7,.5}
\definecolor{lightgrey}{gray}{0.4}
\definecolor{gray}{gray}{0.50}
\lstset{
  language        = Haskell,
  basicstyle      = \scriptsize\ttfamily,
  keywordstyle    = \color{dkblue},     stringstyle     = \color{red},
  identifierstyle = \color{dkgreen},    commentstyle    = \color{gray},
  showspaces      = false,              showstringspaces= false,
  rulecolor       = \color{gray},       showtabs        = false,
  tabsize         = 8,                  breaklines      = true,
  xleftmargin     = 8pt,                xrightmargin    = 8pt,
  frame           = single,             stepnumber      = 1,
  aboveskip       = 10pt plus 3pt,
  belowskip       = 1pt plus 1pt
}
\lstnewenvironment{code}[0]{}{}
\lstnewenvironment{showCode}[0]{\lstset{numbers=none}}{} % only shown, not compiled

% will the real phi please stand up
\renewcommand{\phi}{\varphi}

\pdfinfoomitdate=1
\pdftrailerid{}
\pdfsuppressptexinfo15

\UseRawInputEncoding

\title{Haskelleene}
\subtitle{a very Haskell implmentation of automata, \\ regular expressions, and Kleene's algorithm}
\author{Liam Chung, Brendan Dufty, Lingyuan Ye}
\date{31st May, 2024}

\AtBeginSection[] {
\begin{frame}{Table of Contents}
\begin{columns}[t]
 \begin{column}{.5\textwidth}
     \tableofcontents[sections={1-4}, currentsection, hideallsubsections]
 \end{column}
 \begin{column}{.5\textwidth}
     \tableofcontents[sections={5-8}, currentsection, hideallsubsections]
 \end{column}
\end{columns}
\end{frame}
}

\begin{document}
\frame{\titlepage}

\begin{frame}{Table of Contents}
\begin{columns}[t]
 \begin{column}{.5\textwidth}
     \tableofcontents[sections={1-4}, hideallsubsections]
 \end{column}
 \begin{column}{.5\textwidth}
     \tableofcontents[sections={5-8}, hideallsubsections]
 \end{column}
\end{columns}
\end{frame}

\section{Introduction}
\begin{frame}{Introduction}
    \begin{quote}
        "Automata are pretty cool."
    \end{quote}
    - Liam Chung
\end{frame}

\section{Automata}

\begin{frame}[fragile]{What is an automaton?}
An basic version of a state machine. It takes inputs from some \emph{alphabet}, moving between \emph{states} that may or may not \emph{accept}. 
  \begin{verbatim}
  data DetAut l s = DA { states :: [s]
                       , accept :: [s]
                       , delta  :: l -> s -> s }

  data NDetAut l s = NA { nstates :: [s]
                        , naccept :: [s]
                        , ndelta :: Maybe l -> s -> [s] }
  \end{verbatim}\pause
  \[\begin{tikzcd}
  	1 & 2^{\star}
  	\arrow["b", from=1-1, to=1-1, loop, in=150, out=210, distance=5mm]
  	\arrow["a", curve={height=-6pt}, from=1-1, to=1-2]
  	\arrow["b", curve={height=-6pt}, from=1-2, to=1-1]
  	\arrow["a", from=1-2, to=1-2, loop, in=330, out=30, distance=5mm]
      \end{tikzcd}\]\pause
    \textbf{How do we encode \texttt{delta}?}
\end{frame}

\begin{frame}[fragile]{Automaton data: \texttt{AutData}}
  A simpler representation of our data:
  \begin{verbatim}
  type TDict l s = [(s, [(Maybe l, s)])]
  data AutData l s = AD { stateData :: [s] 
                        , acceptData :: [s] 
                        , transitionData :: TDict l s }
  \end{verbatim}
  Then, encode this data into our nicer types. \\ \pause
  Need to check if data is safe for DAs! \textbf{How?}
\end{frame}
\begin{frame}{The \texttt{Alphabet}}
\end{frame}
\begin{frame}{Implementing semantics}
\end{frame}

\section{Non-deterministic Automata}
\begin{frame}{What is a non-deterministic automaton?}
\end{frame}
\begin{frame}{Implementing semantics}
\end{frame}
\begin{frame}{The power set construction}
\end{frame}

\section{Regular Expressions}
\begin{frame}{What is a regular expression?}
\end{frame}
\begin{frame}{Implementing semantics}
\end{frame}
\begin{frame}{\texttt{Seq} and \texttt{Star} cases}
\end{frame}

\section{Kleene's Theorem}
\begin{frame}{Kleene's Theorem}
  \begin{theorem}
    The following are equivalent, for a language $l \in \mathcal{P}(X)$:
    \begin{itemize}
      \item it is represented by a state in a finite DA\pause
      \item it is represented by a state in a finite NA\pause
      \item it is represented by a regular expression\pause
    \end{itemize}
    in such a case, the language $l$ is called \textbf{regular}.
  \end{theorem}
\end{frame}

\end{document}
